\documentclass{rapport}

\title{Rapport du miniprojet 1 de \textsc{Recherche Opérationnelle}}

\begin{document}

\maketitle

\section{Première partie - Un problème de production}

\subsection{1. Programme linéaire}

Pour modéliser le problème, nous utiliserons les notations suivantes :
\ \\
$T$ est l'ensemble des instants du problèmes (les semaines de production). $T^*$ est $T$ privé de la semaine $0$. $\mathcal{R}$ est l'ensemble des références qu'il est possible de produire.

Ensuite, les données du problèmes sont : $\forall i \in T$ et $\forall r \in \mathcal{R}$, $s_{i,r}$ le stock de la référence $r$ en semaine $i$ et $x_{i,r}$ la quantité produite en semaine $i$ de la référence $r$. Enfin $d_{i,r}$ est la demande en référence $r$ la semaine $i$ et $c_r$ est le coût du stock de la référence $r$. $Q$ est la quantité maximale produite chaque semaine et $N$ est la quantité maximale de référence produite chaque semaine.
\ \\

On cherche à minimiser le coût total (production et stockage). On cherche donc à minimiser $$\sum_{i \in T^*} \sum_{r \in \mathcal{R}} c_r \times s_{i, r}$$ \\
\ \\
Nos contraintes sont les suivantes : \\
\begin{itemize}
\item \emph{Contraintes de positivité} : À chaque semaine et pour chaque référence, le stock doit être positif (ou nul).  De même, on s'interdit de détruire du stock : on doit donc avoir une production positive (ou nulle) pour chaque référence. On fixe donc $$\forall i \in T; \forall r \in \mathcal{R}; s_{i, r} \in \R^+$$ $$\forall i \in T; \forall r \in \mathcal{R}; x_{i, r} \in \R^+$$

\item À chaque nouvelle semaine, on actualise le stock de la nouvelle semaine avec le stock de la semaine précédente, la production et les références expédiées. On a donc $$\forall i \in T; \forall r \in \mathcal{R}; s_{i,r} = x_{i,r} + s_{i-1,r} - d_{i, r}$$
\item On souhaite satisfaire la demande chaque semaine. On souhaite donc que $\forall i \in T; \forall r \in \mathcal{R}; x_{i,r} + s_{i-1,r} \ge d_{i, r}$. Or, sachant que les stocks sont toujours positifs, cette contrainte est inclue dans la contrainte précédente : on n'a donc pas besoin de rajouter de contrainte supplémentaire dans le modèle.
\item D'autre part, on ne peut pas produire plus de $N$ références par semaine, ce qu'on peut traduire par : $$\forall i \in T^*; \sum_{r \in \mathcal{R}} \mathbf{1}_{x_{i, r} > 0} \le N$$.
\item Enfin, il faut satisfaire la capacité de production de l'usine : On ajoute donc la relation :
$$\forall i \in T^*; \sum_{r \in \mathcal{R}} x_{i, r} \le Q$$
\end{itemize}

Notre problème se modélise donc par programme linéaire suivant : \\
\ \\

\begin{equation*}
\begin{aligned}
& \min \ \ \  \sum_{i \in T^*} \sum_{r \in \mathcal{R}} c_r \times s_{i, r} \\
& \text{sous contraintes : } \\
\ \\
& \forall i \in T; \forall r \in \mathcal{R}; s_{i,r} = x_{i,r} + s_{i-1,r} - d_{i, r}\\
& \forall i \in T^*; \sum_{r \in \mathcal{R}} \mathbf{1}_{x_{i, r} > 0} \le N \\
& \forall i \in T^*; \sum_{r \in \mathcal{R}} x_{i, r} \le Q \\
& \forall i \in T; \forall r \in \mathcal{R}; s_{i, r} \in \R^+ \\
& \forall i \in T; \forall r \in \mathcal{R}; x_{i, r} \in \R^+
\end{aligned}
\end{equation*}
\ \\
\ \\
\emph{Détails d'implémentation} : \\
Pour traduire la condition avec l'indicatrice $\forall i \in T^*; \sum_{r \in \mathcal{R}} \mathbf{1}_{x_{i, r} > 0} \le N$, on introduit la matrice A, dont l'élément (i,r) vaut 1 si la référence r a été produite pendant la semaine i, et zéro sinon. Il faut donc ajouter une contrainte qui forcera à produire les uniquement les références mentionnées dans la matrice A. \\
\ \\
On traduit donc la 2e contrainte par :
$$\forall i \in T^*, \sum_{r \in \mathcal{R}}a_{i,r} \le N $$
$$\forall r \in \mathcal{R} $$

Par ailleurs, dans l'implémentation, on inverse les lignes et colonnes de d (conformément à la structure des données reçues par mail)

\subsection{2. Résolution avec GLPK}
On récupère les trois sets de données envoyés par mails, et on les place dans les fichiers \emph{Q1\_dataset\_1.dat}, \emph{Q1\_dataset\_2.dat} et \emph{Q1\_dataset\_3.dat}. \\
Le programme linéaire a été traduit dans GLPK dans le fichier \emph{production.mod}.
\ \\
\subsubsection{Première instance}
Le programme ne trouve pas de solution réalisable pour la première instance. \ \\
En prenant $N'=3$ (sachant que $N=2$) et $Q'=Q$, on trouve une solution réalisable, de coût 226. $Q+N$ minore la solution optimale mais n'est pas réalisable, $Q'+N'=Q+N+1$. Cette instance étant réalisable et s'agissant un problème linéaire en nombres entiers, c'est la solution optimale. \\

En prenant donc $Q'=Q$ et $N'=N+1=3$, on trouve le plan de production suivant (la matrice se lit avec la ligne r = référence r, colonne i = semaine i, conformément au format des données reçues) :

\[x=
\begin{matrix}
0 & 5 & 0 & 6 & 0 & 4 \\
2 & 0 & 8 & 0 & 0 & 0 \\
0 & 8 & 0 & 0 & 5 & 0 \\
3 & 0 & 0 & 7 & 0 & 2 \\
0 & 0 & 7 & 0 & 0 & 2 \\
0 & 0 & 5 & 0 & 0 & 0 \\
0 & 5 & 0 & 0 & 8 & 0 \\
0 & 0 & 0 & 0 & 5 & 0 \\
5 & 0 & 0 & 7 & 0 & 0
\end{matrix}
\]

avec un coût total de 226. \\
Si on veut satisfaire la demande, il vaut donc mieux ouvrir une nouvelle ligne de production afin d'être capable de produire une référence supplémentaire dans la semaine (et ainsi augmenter $N$).

\subsubsection{Deuxième instance}
On trouve le plan optimal de production suivant (on a toujours $x[r,i]$ la quantité produite de la référence r à la semaine i, conformément au format des données reçues):

\[x=
\begin{matrix}
0 & 0 & 2 & 5 & 7 & 4 & 5 & 0 & 0 & 3 \\
0 & 14 & 1 & 0 & 0 & 0 & 0 & 0 & 0 & 0 \\
0 & 0 & 14 & 0 & 0 & 0 & 0 & 8 & 0 & 6 \\
0 & 0 & 0 & 7 & 1 & 2 & 1 & 3 & 3 & 2 \\
0 & 0 & 0 & 4 & 0 & 2 & 4 & 4 & 5 & 0 \\
15 & 0 & 0 & 0 & 0 & 0 & 0 & 0 & 0 & 0 \\
0 & 0 & 0 & 0 & 8 & 0 & 0 & 2 & 0 & 0 \\
0 & 0 & 0 & 0 & 0 & 4 & 0 & 2 & 3 & 0 \\
0 & 3 & 6 & 0 & 2 & 0 & 4 & 0 & 3 & 0 \\
0 & 0 & 0 & 4 & 2 & 0 & 6 & 0 & 2 & 5 \\
8 & 6 & 0 & 0 & 0 & 8 & 0 & 0 & 0 & 4 \\
0 & 0 & 0 & 3 & 3 & 3 & 3 & 4 & 7 & 3 \\
\end{matrix}
\]

avec un coût total de 766.

\subsubsection{Troisième instance}
On trouve le plan optimal de production suivant (on a toujours $x[r,i]$ la quantité produite de la référence r à la semaine i, conformément au format des données reçues):

\[x=
\begin{matrix}
0 & 18 & 0 & 0 & 0 & 8 & 3 & 0 & 0 & 5 \\
0 & 0 & 0 & 6 & 6 & 0 & 0 & 4 & 0 & 0 \\
0 & 0 & 9 & 0 & 3 & 2 & 4 & 0 & 3 & 7 \\
0 & 1 & 10 & 0 & 0 & 3 & 0 & 3 & 5 & 0 \\
0 & 1 & 2 & 6 & 1 & 2 & 5 & 5 & 5 & 0 \\
14 & 2 & 0 & 0 & 0 & 0 & 0 & 0 & 0 & 0 \\
0 & 0 & 0 & 0 & 8 & 0 & 0 & 0 & 0 & 1 \\
0 & 0 & 0 & 0 & 0 & 3 & 1 & 1 & 4 & 0 \\
1 & 4 & 1 & 8 & 2 & 0 & 0 & 1 & 1 & 2 \\
0 & 1 & 5 & 0 & 4 & 0 & 6 & 0 & 2 & 5 \\
0 & 0 & 0 & 4 & 0 & 6 & 6 & 8 & 0 & 4 \\
0 & 0 & 0 & 3 & 3 & 3 & 2 & 5 & 7 & 3 \\
\end{matrix}
\]

avec un coût total de 368.

\section{Deuxième partie - Un problème d'ordonnancement de trains et de navettes}

Nous avons décider de formuler le problème sous forme de programme linéaire en nombre entier.

\subsection{Notation utilisée}

Les paramètres du problèmes sont :$e$, $h$ et $p$ sont respectivement le nombre d'eurostars, de HGV et de PAX que l'on souhaite faire passer dans le tunnel. $n = e + h + p$ et $n_{mat} = [e, h, p]$. Enfin, $t$ est le tableau des tampons comme défini dans l'ennoncé.

On introduit également la quantité $f_{max} = \frac{3600}{n - 1}$ qui correspond à une borne supérieure facile à calculer de la flexibilité.

Les variables que l'on introduit pour résoudre le problème sont :
\begin{itemize}
	\item $T$ est une matrice $n \times 3$ qui permet de lire l'ordre des trains qui partent. Ainsi, le coefficient $T_{i,j}$ vaut $1$ si un train de type $j$ est le $i$\up{ième} mobile à entrer dans le tunnel. Il vaut 0 sinon.
	\item $D$ est une matrice $n \times 3$ qui donne la date de départ des trains. Là où $T$ est nulle, $D$ est nulle. Sinon, la valeur de $D_{i, j}$ est le moment (en seconde) où le train part du quai. (Notons que le premier départ possible est à $1s$ et le dernier à $3600s$ au lieu de $0s$ et $3599s$).
	\item $E$ est une matrice $n \times n$ qui va montrer les moments de départs des eurostars de la première demi-heure. Ainsi, $E_{i, j}$ vaut $1$ si le $i$\up{ième} train à partir est un eurostar, le $j$\up{ième} train est également un eurostar et qu'ils sont séparés de $1800s$. Cela équivaut à $\mathbf{1_{T_{i,1} = T_{j,1} = 1\ \cup\ D_{j,1} - D_{i,1} = 1800}}$.
	\item $S$ est un vecteur à $n$ lignes. Il vaut $1$ si le train partant en $i$\up{ième} position est un eurostar et que la date de départ est inférieure à $1800s$.
	\item Finalement, $f_{value}$ est la valeur de la flexibilité.
\end{itemize}

\subsection{Représentation linéaire des contraintes}

Notre fonction objectif est donc simplement $f_{value}$ qu'on cherche à maximiser.

\begin{itemize}
	\item Il n'y a qu'un seul train qui part à chaque horaire : $\forall i \in [n]; \ \sum_{j = 1}^3 T_{i, j} = 1$.
	\item On envoie le bon nombre de chaque type de train : $\forall j \in [3]; \ \sum_{j = 1}^n T_{i, j} = n_{mat}[j]$.
	\item On contraint $D$ à être nulle quand $T$ est nulle et sinon, à être entre $1$ et $3600$ : $\forall i \in [n], \forall j \in [3]; \ T_{i, j} \le D_{i, j} \le 3600 * T_{i, j}$.
	\item Cette contrainte représente la contrainte d'écart minimal entre deux trains consécutifs : 
	Si on considère dans un premier temps les trains qui se suivent dans la même heure, on a une contrainte du type : $D_{k, l} - D_{i, j} \ge t_{j, l} + f_{value}$ (sans se soucier des indices pour le moment). Si les trains se suivent sur deux horaires différentes on modifierais simplement de telle sorte : $D_{k, l} - (D_{i, j} - 3600) \ge t_{j, l} + f_{value}$. Maintenant, lorsque l'on parcourt les indices, il y a des cas où $D$ sera nulle, et donc il faut qu'on impose rien dans ce cas. C'est pourquoi on soustrait au terme de droite la valeur : $(3600 + 480 + f_{max}) \times ((1 - T_{i, j}) + (1 - T_{k, l}))$ qui vaut 0 lorsque $D_{i, j}$ et $D_{k, l}$ sont non nuls, et est plus grand que $t_{j, l} + f_{value}$ sinon. En rajoutant ces conditions, les indices peuvent librement parcourir $[n]$ et $[3]$.
	\begin{itemize}
		\item $\forall i \in [n-1], \forall k \in \llbracket i, n \rrbracket, \forall (j, l) \in [3]^2; D_{k, l} - D_{i, j} \ge t_{j, l} + f_{value} - (3600 + 480 + f_{max}) \times ((1 - T_{i, j}) + (1 - T_{k, l}))$
		\item $\forall i \in [n-1], \forall k \in \llbracket i, n \rrbracket, \forall (j, l) \in [3]^2; D_{k, l} - (D_{i, j} - 3600) \ge t_{j, l} + f_{value} - (3600 + 480 + f_{max}) \times ((1 - T_{i, j}) + (1 - T_{k, l}))$
	\end{itemize}
	\item Cette contrainte est la première de celle représentant l'écart de 30 minutes entre deux eurostars. Elle assure que $E$ ne donne des informations que sur les eurostars de la première demi-heure. $\forall i \in [n], \forall j \in [i]; E_{i, j} = 0$.
	\item On s'assure ici qu'il existe un unique eurostar associé à un autre eurostar.
	\begin{itemize}
		\item $\forall i \in [n-1]; \sum_{j = i+1}^n E_{i, j} = S_i$.
		\item $\forall j \in \llbracket 2, n \rrbracket; \sum_{i = 1}^{j-1} E_{i, j} = T_{i, 1} - S_i$.
	\end{itemize}
	\item Enfin, cette dernière contrainte fixe l'écart, lorsqu'il est justifiée à 30 minutes : $\forall i \in [n-1], \forall j \in \llbracket i+1, n \rrbracket; - 5400 * (1 - E_{i, j})  \le D_{j, 1} - D_{i, 1} - 1800 \le 1800 * (1 - E_{i, j})$

\end{itemize}

En lançant plusieurs fois notre programme linéaire, celui-ci propose plusieurs grilles horaires, comme par exemple :

\[Grille \  horaire_1 =
\begin{matrix}
H1 & H2 & H3 & H4 & E1 & E2 & P1 \\
1 & 224 & 447 & 670 & 1192 & 1415 & 1578 \\
\ \\
H5 & H6 & H7 & H8 & E3 & E4 & P2 \\
1801 & 2024 & 2247 & 2470 & 2993 & 3215 & 3378
\end{matrix}
\]

\[Grille \  horaire_2 =
\begin{matrix}
H1 & H2 & E1 & E2 & P1 & H3 & H4 \\
1 & 224 & 747 & 970 & 1132 & 1355 & 1578 \\
\ \\
H5 & H6 & H7 & H8 & E3 & E4 & P2 \\
1801 & 2024 & 2547 & 2770 & 2932 & 3155 & 3378
\end{matrix}
\]


Qui ont toutes une flexibilité de 42 secondes. Comme les grilles horaires correspondent à une solution optimale d'un programme linéaire en nombres entiers, la flexibilité trouvée est donc optimale.

\subsection{2. Le plus de navettes HGV possibles}
On augmente h dans \emph{Q2\_data\_1.dat}. À partir de $h=12$, le solveur indique qu'aucune solution réalisable n'existe : $h=11$ est donc le maximum de navettes HGV qu'on puisse faire passer pour $e=4$ et $p=2$.

Pour des grandes valeurs de $h$, le programme met beaucoup de temps à calculer, c'est pourquoi nous posons la flexibilité à zéro : on souhaite simplement trouver une solution réalisable, pas nécessairement la grille avec la flexibilité la plus grande possible. \\
Pour $h=11$, on peut proposer la grille horaire suivante :


\[Grille \  horaire_2 =
\begin{matrix}
H1 & H2 & H3 & H4 & E1 & E2 & P1 & P2 \\
1 & 181 & 361 & 541 & 1021 & 1201 & 1321 & 1441\\
\ \\
H5 & H6 & H7 & H8 & H9 & E3 & E4 & H10 & H11 \\
1621 & 1801 & 1981 & 2161 & 2341 & 2821 & 3001 & 3241 & 3421
\end{matrix}
\]


\end{document}
