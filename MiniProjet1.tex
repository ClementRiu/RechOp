\documentclass{rapport}

\title{Rapport du miniprojet 1 de \textsc{Recherche Opérationnelle}}

\begin{document}

\maketitle

\section{2/ Problème de production}

\subsection{1. Programme linéaire}

Pour modéliser le problème nous allons utiliser les notations suivantes :

$T$ est l'ensemble des instant du problèmes (les semaines de production). $T^*$ est $T$ privé de la semaine $0$. $\mathcal{R}$ est l'ensemble des références qu'il est possible de produire.

Ensuite, les données du problèmes seront : $\forall t \in T$ et $\forall r \in \mathcal{R}$, $s_{t,r}$ le stock de la référence $r$ en semaine $t$ et $x_{t,r}$ la quantité produite en semaine $t$ de la référence $r$. Enfin $d_{t,r}$ est la demande en référence $r$ la semaine $t$ et $c_r$ est le coût du stock de la référence $r$. $Q$ est la quantité maximale produite chaque semaine et $N$ est la quantité maximale de référence produite chaque semaine.

Notre programme linéaire sera le suivant :

\begin{equation*}
\begin{aligned}
& \min \ \ \  \sum_{t \in T^*} \sum_{r \in \mathcal{R}} c_r \times s_{t, r} \\
& \text{sous contraintes : } \\
& \forall t \in T; \forall r \in \mathcal{R}; s_{t,r} = x_{t,r} + s_{t-1,r} - d_{t, r}\\
& \forall t \in T^*; \sum_{r \in \mathcal{R}} \mathbf{1}_{x_{t, r} > 0} \le N \\
& \forall t \in T^*; \sum_{r \in \mathcal{R}} x_{t, r} \le Q \\
& \forall t \in T; \forall r \in \mathcal{R}; s_{t, r} \in \R^+ \\
& \forall t \in T; \forall r \in \mathcal{R}; x_{t, r} \in \R^+
\end{aligned}
\end{equation*}



\end{document}
