\documentclass{rapport}

\title{Rapport du miniprojet 1 de \textsc{Recherche Opérationnelle}}

\begin{document}

\maketitle

\section{Première partie - Un problème de production}

\subsection{1. Programme linéaire}

Pour modéliser le problème nous allons utiliser les notations suivantes :

$T$ est l'ensemble des instant du problèmes (les semaines de production). $T^*$ est $T$ privé de la semaine $0$. $\mathcal{R}$ est l'ensemble des références qu'il est possible de produire.

Ensuite, les données du problèmes seront : $\forall i \in T$ et $\forall r \in \mathcal{R}$, $s_{i,r}$ le stock de la référence $r$ en semaine $i$ et $x_{i,r}$ la quantité produite en semaine $i$ de la référence $r$. Enfin $d_{i,r}$ est la demande en référence $r$ la semaine $i$ et $c_r$ est le coût du stock de la référence $r$. $Q$ est la quantité maximale produite chaque semaine et $N$ est la quantité maximale de référence produite chaque semaine.


On cherche à minimiser le coût total (production et stockage). On cherche donc à minimiser $\sum_{i \in T^*} \sum_{r \in \mathcal{R}} c_r \times s_{i, r}$


Notre programme linéaire sera le suivant : \\

\begin{equation*}
\begin{aligned}
& \min \ \ \  \sum_{i \in T^*} \sum_{r \in \mathcal{R}} c_r \times s_{i, r} \\
& \text{sous contraintes : } \\
& \forall i \in T; \forall r \in \mathcal{R}; s_{i,r} = x_{i,r} + s_{i-1,r} - d_{i, r}\\
& \forall i \in T^*; \sum_{r \in \mathcal{R}} \mathbf{1}_{x_{i, r} > 0} \le N \\
& \forall i \in T^*; \sum_{r \in \mathcal{R}} x_{i, r} \le Q \\
& \forall i \in T; \forall r \in \mathcal{R}; s_{i, r} \in \R^+ \\
& \forall i \in T; \forall r \in \mathcal{R}; x_{i, r} \in \R^+
\end{aligned}
\end{equation*}

\begin{lstlisting}
caca
\end{lstlisting}

\end{document}
