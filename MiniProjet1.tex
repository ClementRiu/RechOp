\documentclass{rapport}

\title{Rapport du miniprojet 1 de \textsc{Recherche Opérationnelle}}

\begin{document}

\maketitle

\section{Première partie - Un problème de production}

\subsection{1. Programme linéaire}

Pour modéliser le problème, nous utiliserons les notations suivantes :
\ \\
$T$ est l'ensemble des instants du problèmes (les semaines de production). $T^*$ est $T$ privé de la semaine $0$. $\mathcal{R}$ est l'ensemble des références qu'il est possible de produire.

Ensuite, les données du problèmes sont : $\forall i \in T$ et $\forall r \in \mathcal{R}$, $s_{i,r}$ le stock de la référence $r$ en semaine $i$ et $x_{i,r}$ la quantité produite en semaine $i$ de la référence $r$. Enfin $d_{i,r}$ est la demande en référence $r$ la semaine $i$ et $c_r$ est le coût du stock de la référence $r$. $Q$ est la quantité maximale produite chaque semaine et $N$ est la quantité maximale de référence produite chaque semaine.
\ \\

On cherche à minimiser le coût total (production et stockage). On cherche donc à minimiser $$\sum_{i \in T^*} \sum_{r \in \mathcal{R}} c_r \times s_{i, r}$$ \\
\ \\
Nos contraintes sont les suivantes : \\
\begin{itemize}
\item \emph{Contraintes de positivité} : À chaque semaine et pour chaque référence, le stock doit être positif (ou nul).  De même, on s'interdit de détruire du stock : on doit donc avoir une production positive (ou nulle) pour chaque référence. On fixe donc $$\forall i \in T; \forall r \in \mathcal{R}; s_{i, r} \in \R^+$$ $$\forall i \in T; \forall r \in \mathcal{R}; x_{i, r} \in \R^+$$

\item À chaque nouvelle semaine, on actualise le stock de la nouvelle semaine avec le stock de la semaine précédente, la production et les références expédiées. On a donc $$\forall i \in T; \forall r \in \mathcal{R}; s_{i,r} = x_{i,r} + s_{i-1,r} - d_{i, r}$$
\item On souhaite satisfaire la demande chaque semaine. On souhaite donc que $\forall i \in T; \forall r \in \mathcal{R}; x_{i,r} + s_{i-1,r} \ge d_{i, r}$. Or, sachant que les stocks sont toujours positifs, cette contrainte est inclue dans la contrainte précédente : on n'a donc pas besoin de rajouter de contrainte supplémentaire dans le modèle.
\item D'autre part, on ne peut pas produire plus de $N$ références par semaine, ce qu'on peut traduire par : $$\forall i \in T^*; \sum_{r \in \mathcal{R}} \mathbf{1}_{x_{i, r} > 0} \le N$$.
\item Enfin, il faut satisfaire la capacité de production de l'usine : On ajoute donc la relation :
$$\forall i \in T^*; \sum_{r \in \mathcal{R}} x_{i, r} \le Q$$
\end{itemize}

Notre problème se modélise donc par programme linéaire suivant : \\
\ \\

\begin{equation*}
\begin{aligned}
& \min \ \ \  \sum_{i \in T^*} \sum_{r \in \mathcal{R}} c_r \times s_{i, r} \\
& \text{sous contraintes : } \\
\ \\
& \forall i \in T; \forall r \in \mathcal{R}; s_{i,r} = x_{i,r} + s_{i-1,r} - d_{i, r}\\
& \forall i \in T^*; \sum_{r \in \mathcal{R}} \mathbf{1}_{x_{i, r} > 0} \le N \\
& \forall i \in T^*; \sum_{r \in \mathcal{R}} x_{i, r} \le Q \\
& \forall i \in T; \forall r \in \mathcal{R}; s_{i, r} \in \R^+ \\
& \forall i \in T; \forall r \in \mathcal{R}; x_{i, r} \in \R^+
\end{aligned}
\end{equation*}
\ \\
\ \\
\emph{Détails d'implémentation} : \\
Pour traduire la condition avec l'indicatrice $\forall i \in T^*; \sum_{r \in \mathcal{R}} \mathbf{1}_{x_{i, r} > 0} \le N$, on introduit la matrice A, dont l'élément (i,r) vaut 1 si la référence r a été produite pendant la semaine i, et zéro sinon. Il faut donc ajouter une contrainte qui forcera à produire les uniquement les références mentionnées dans la matrice A. \\
\ \\
On traduit donc la 2e contrainte par :
$$\forall i \in T^*, \sum_{r \in \mathcal{R}}a_{i,r} \le N $$
$$\forall r \in \mathcal{R} $$

Par ailleurs, dans l'implémentation, on inverse les lignes et colonnes de d (conformément à la structure des données reçues par mail)

\subsection{2. Résolution avec GLPK}
On récupère les trois sets de données envoyés par mails, et on les place dans les fichiers \emph{Q1\_dataset\_1.dat}, \emph{Q1\_dataset\_2.dat} et \emph{Q1\_dataset\_3.dat}. \\
Le programme linéaire a été traduit dans GLPK dans le fichier \emph{production.mod}.
\ \\
\subsubsection{Première instance}
Le programme ne trouve pas de solution réalisable pour la première instance. \ \\
En prenant $N'=3$ (sachant que $N=2$) et $Q'=Q$, on trouve une solution réalisable, de coût 226. $Q+N$ minore la solution optimale mais n'est pas réalisable, $Q'+N'=Q+N+1$. Cette instance étant réalisable et s'agissant un problème linéaire en nombres entiers, c'est la solution optimale. \\

En prenant donc $Q'=Q$ et $N'=N+1=3$, on trouve le plan de production suivant (la matrice se lit avec la ligne r = référence r, colonne i = semaine i, conformément au format des données reçues) :

\[x=
\begin{matrix}
0 & 5 & 0 & 6 & 0 & 4 \\
2 & 0 & 8 & 0 & 0 & 0 \\
0 & 8 & 0 & 0 & 5 & 0 \\
3 & 0 & 0 & 7 & 0 & 2 \\
0 & 0 & 7 & 0 & 0 & 2 \\
0 & 0 & 5 & 0 & 0 & 0 \\
0 & 5 & 0 & 0 & 8 & 0 \\
0 & 0 & 0 & 0 & 5 & 0 \\
5 & 0 & 0 & 7 & 0 & 0
\end{matrix}
\]

avec un coût total de 226. \\
Si on veut satisfaire la demande, il vaut donc mieux ouvrir une nouvelle ligne de production afin d'être capable de produire une référence supplémentaire dans la semaine (et ainsi augmenter $N$).

\subsubsection{Deuxième instance}
On trouve le plan optimal de production suivant (on a toujours $x[r,i]$ la quantité produite de la référence r à la semaine i, conformément au format des données reçues):

\[x=
\begin{matrix}
0 & 0 & 2 & 5 & 7 & 4 & 5 & 0 & 0 & 3 \\
0 & 14 & 1 & 0 & 0 & 0 & 0 & 0 & 0 & 0 \\
0 & 0 & 14 & 0 & 0 & 0 & 0 & 8 & 0 & 6 \\
0 & 0 & 0 & 7 & 1 & 2 & 1 & 3 & 3 & 2 \\
0 & 0 & 0 & 4 & 0 & 2 & 4 & 4 & 5 & 0 \\
15 & 0 & 0 & 0 & 0 & 0 & 0 & 0 & 0 & 0 \\
0 & 0 & 0 & 0 & 8 & 0 & 0 & 2 & 0 & 0 \\
0 & 0 & 0 & 0 & 0 & 4 & 0 & 2 & 3 & 0 \\
0 & 3 & 6 & 0 & 2 & 0 & 4 & 0 & 3 & 0 \\
0 & 0 & 0 & 4 & 2 & 0 & 6 & 0 & 2 & 5 \\
8 & 6 & 0 & 0 & 0 & 8 & 0 & 0 & 0 & 4 \\
0 & 0 & 0 & 3 & 3 & 3 & 3 & 4 & 7 & 3 \\
\end{matrix}
\]

avec un coût total de 766.

\subsubsection{Troisième instance}
On trouve le plan optimal de production suivant (on a toujours $x[r,i]$ la quantité produite de la référence r à la semaine i, conformément au format des données reçues):

\[x=
\begin{matrix}
0 & 18 & 0 & 0 & 0 & 8 & 3 & 0 & 0 & 5 \\
0 & 0 & 0 & 6 & 6 & 0 & 0 & 4 & 0 & 0 \\
0 & 0 & 9 & 0 & 3 & 2 & 4 & 0 & 3 & 7 \\
0 & 1 & 10 & 0 & 0 & 3 & 0 & 3 & 5 & 0 \\
0 & 1 & 2 & 6 & 1 & 2 & 5 & 5 & 5 & 0 \\
14 & 2 & 0 & 0 & 0 & 0 & 0 & 0 & 0 & 0 \\
0 & 0 & 0 & 0 & 8 & 0 & 0 & 0 & 0 & 1 \\
0 & 0 & 0 & 0 & 0 & 3 & 1 & 1 & 4 & 0 \\
1 & 4 & 1 & 8 & 2 & 0 & 0 & 1 & 1 & 2 \\
0 & 1 & 5 & 0 & 4 & 0 & 6 & 0 & 2 & 5 \\
0 & 0 & 0 & 4 & 0 & 6 & 6 & 8 & 0 & 4 \\
0 & 0 & 0 & 3 & 3 & 3 & 2 & 5 & 7 & 3 \\
\end{matrix}
\]

avec un coût total de 368.

\section{Deuxième partie - Un problème d'ordonnancement de trains et de navettes}


\end{document}
